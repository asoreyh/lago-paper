\section{Science case}\label{sec:science}

%%Fig. \ref{fig:lago-fb}, shows the comparison between a neutron monitor
%%in Rome, and a single WCD of 1.8 m$^2$ in Bariloche \cite{Asorey2013b} measuring
%%a Forbush decrease in 2012. As can be seen the LAGO detector shows similar
%%relative variation as the other two measurements.
%%
%%explayarse mas sobre space wheather con lago
%%
%%\begin{figure}
%%\includegraphics[width=.9\textwidth]{images/icrc2013-1109-05.pdf}
%%\caption{A Forbush decrease event seen by a single LAGO detector. The
%%comparison between the WCD in Bariloche and a neutron monitor in Rome shows a very good agreement.}
%%\label{fig:lago-fb}
%%\end{figure}

%%%%%%%%%%%%%%%%%%<< aca empieza la parte de dasso >>%%%%%%%%%%%%%%%%%%%%%%%%%%%%%%%%%%%%%%%%

% Intro context (que problema y a quien le interesa).
The transport of energetic particles or cosmic rays (CRs, both solar and galactic origin) in the
interplanetary medium and in the terrestrial environment (magnetosphere and upper/middle/lower atmosphere)
is a major topic of interest for space weather, presenting several unsolved questions
\cite[e.g., ][]{Wibberenz98, Belov00, Smart09, Kudela13}.
% Large scaler and transients.
Different physical mechanisms modulate the transport of CRs in the heliosphere.
Generally, they can be divided into large scale processes (mainly related to the magnetic field configuration in the whole heliosphere) and transient phenomena.
The last ones are mainly produced by transient solar processes, such as Interplanetary Coronal Mass Ejections (ICMEs)
or interplanetary shocks \cite[e.g., ][]{Wibberenz98}.

% NMs as historically typical ground detectors of CRs
Since more than 50 decades, neutron monitors have been the usual ground detectors for recording indirect 
observations of the variability of primary CRs fluxes arriving Earth with energies of interest
for space physics \cite[e.g., ][]{Meyer55}.
These detectos have provided data and clues to better understand different physical mechanisms
occurring during the interplanetary transport of CRs,
such as the anticorrelation between the solar activity and the galactic CRs fluxes,
the solar rotation combined with plasma streams ejected from coronal holes producing 27 days recurring
intensity variations, and transient Forbush decrease events (Fds).
Forbush decreases are observed depressions of the CR flux at the Earth ground level, which can last from hours to weeks.
They are generally observed in connection with the presence of ICMEs in the vicinity of the Earth \cite[e.g., ][]{Cane00}.

% muon detectors 
Muon detectors and muon telescopes are newer ground observatories for similar, and a bit larger, ranges of energies than neutron monitors
that can record variability of CRs associated with space physics and space weather phenomena \cite[e.g., ][]{Munakata12}.

% Recent CWDs (Auger and LAGO)
Recently, water Cherenkov detectors (WCDs) have shown the capacity of observing the variability
of the flux of CRs related with space weather phenomena \cite{FALAugerscalers11, AugerCOLAGE12, ICRC-Auger, ICRC-LAGO}.
The Pierre Auger Observatory, located in Malarg\"ue, Argentina (69.3 o W, 35.3 o S, 1400 m a.s.l.), can observe the variability
of CRs during a Forbus decrease using a single particle technique,
which consists in recording low threshold rates (the so-called scaler mode)
with all the surface detectors of its array \cite{FALAugerscalers11}.

% Some details of scaler and histogram modes at PAO.
The scaler mode at the Pierre Auger Observatory can detect a high total count rate, coming from the large area of the surface detectors
(greater than 16,000 m$^2$), reaching a flux of secondary particles as high as $\sim 10^8$ counts per minute
with a median energy for primary particles of $\sim 90$ GeV \cite{AugerCOLAGE12}, providing a very low statistical uncertainty for this counting.
From a comparison with Neutron Monitors from the Observatory at Los Cerrillos (6NM64), Chile, located
$\sim 250$ km westward from Malarg\"ue, it was shown the excellent sensitivity also to the daily modulation of the scalers \cite{AugerCOLAGE12}.
Another mode of interest developed at the Pierre Auger Observatory is the 'histogram mode', in which
the number of secundary particles are counted from discrimination of the deposited energy.
Forbush decreased has been observed even at energies as larger as XXX [ICRC-Auger-China].

% LAGO mostro que puede observar Forbush ICRC-LAGO-Rio
{\bf [Esto se debe compatibilizar con la intro del paper]}
The LatinAmerican Giant Observatory (LAGO) is a non-centralized network composed by more than 30 institutions from nine Latin American countries.
It is an extended CR observatory formed by WCDs, that make observations of the temporal evolution of the radiation flux at ground level.
{\bf [Decir que la red (citar a la otra seccion) tiene/tendra detectores a lo largo de una banda de longitudes (entre tal y tal)
que recorre un gran rango de latitudes (cuantificar).]}

% what to do in this section. Caso Study.
As a simple example of the potentiality of LAGO, in this section we describe
the interplanetary manifestation of a solar ejection happened on March 2012,
and the LAGO observations of the associated Fd.

% Contar un poco la historia del sol a la tierra, las particulas observadas en ese periodo.
Several Coronal Mass Ejections were erupted from the NOAA-AR-11429 during March 2012.
Three of them were launched toward Earth and produced space weather effects in the terrestrial ambient.
In particular, the CME ejected at 00:15 UT of March 7 was the largest one and was associated
with a long duration X5.4 flare peaked at 00:24 UT, launched with velocities $\sim$ 2400 km s$^{-1}$ \cite{Liu13}.
Depending on the velocity and the magentic polarity of ICMEs arriving Earth, they can produce geomagnetic storm,
enhancing the activity of the ring current system in the terrestrial environment
with decay times of the order of several hours \cite{Dasso02}.
The arrival of this ICME with its driven shock to Earth occurred at 10:19 UT of March 8,
and produced a major geomagnetic storm with consequence on the index Dst,
which reached a significant negative value of Dst=-143 nT.
This ICME also had strong consequences on the Earth CRs arrival. Neutron monitors present significant decrease
of their countings, observing a clear Forbush decrease in time agreement with the arrival of the ICME,
in particular the South Pole Neutron Monitor observes a decrease of $\sim -15\%$ \cite{DiFino14}.
%* Produce un Forbush observado por varios NMs (citar paper de Di Fino). Cuantificar porcentaje.
% La fig 4 del paper de ISS [Di Fino et al., J. Space Weather Space Clim, 2014] muestra el Forbush para dif NMs. 
% En South Pole (SOPO) alcanza mas de 15%. 
% Hay paper de NMs para este dia? Sip, el paper de ISS muestra una figura del Fd obs por varios NMs.

% Decir que este Forbush fue observado en primarias por PAMELA [citar a COSPAR poster] ?

% Contar que LAGO observo el Forbush.
% Comparacion con NM de Roma.
% (Hay que mencionar algun acknowledgment por mostrar resultados de Roma?).
Figure xxx shows this Forbush decrease observed by a single WCD of LAGO in Bariloche with a surface of 1.8 m$^2$, Argentina.
It can be osberved that this LAGO detector observed a FD peak of almost $\sim - 6\%$ at a place with a rigidity cutoff of XXX.
For a comparison, this figure also shows observations from the Neutron Monitor in Rome [cita?, agradecimientos?], where can be
seen the very good agreement between LAGO and Rome observations.

{\bf [Se sabe que hubo otra ICME que llego a la Tierra aprox mediodia del 12 de Marzo, con la ICME llegando hacia el final del 12/mar.
En Roma se ve un segundo Forbush montado antes que termine el primer. Pero parece que en LAGO hay un gap justo alli. Decimos algo del 2do evento?.
Tambien se ve que el final del Forbush ocurre aprox 23 marzo. Decimos algo sobre donde estan las ICMEs ese dia o se hace muy largo?
Creo que hay un paper que menciona algo de esto.]}
% * Mas alla de la Tierra?  Liu Apj 2014.

{\bf [Esto podria repetirse, reescribiendolo, en las conclusiones del paper]}
One of the major advantages of LAGO is its extension in geographical latitude (from equatorial latitudes up to the antartic region) and
altitudes (from sea level up to more than 5000 meters over sea level). Thus, the net covers a huge range of geomagnetic rigidity cut-offs
and atmospheric absorption/reaction levels.
From combining different ground sites, observations from the LAGO network, will make possible studies of the heliospheric modulation of galactic cosmic rays,
in particular will allow to study long-term modulation as well as transient events.

