\documentclass{article}
\begin{document}
Detector \emph{“Chimbito”} de la colaboración LAGO Ecuador \emph{blank line}
Dimensiones y \emp{HW} del detector
Tanque de 1100 litros de tipo comercial
1m de diámetro, 1,4m de altura	
PMT \emph{Photonis} XP1812 de 9”
Electrónica LAGO de Bariloche
SW LAGO versión v2r0
Histograma de carga global de 824 horas de datos \emph{blank line}
Pulso Típico de 824 horas de datos \emph{blank line}
Detalle del sitio \emph{blank line}
Ubicación: 	Escuela de Física - Escuela Superior Politécnica de Chimborazo
Riobamba - Ecuador
Altura		2870 m.s.n.m
Latitud		1º 39’ 58’’ S
Longitud:	78º 39’ 33’’ O
Rigidity cut off	15.02 GeV \emph{blank line}
Estado
Actualmente operativo, tomando datos continuos desde el 17 de Julio 2013 \emph{blank line}
Planes a futuro
Instalación de un detector con Electrónica LAGO de Bariloche y PMT Photonis de 9” en el primer refugio del Chimborazo (aprox 4500 m.s.n.m)
Instalación de dos tanques pequeños para estudios de física solar con electrónica por determinar y PMT de 5” en la ciudad de Quito
\end{document}
