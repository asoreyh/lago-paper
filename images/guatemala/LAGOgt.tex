\documentclass{article}
\begin{document}
\begin{center}
{\bf LAGO Guatemala Site}
\end{center}

\bigskip
\bigskip
\begin{enumerate}

\item In Guatemala there is a WCD prototype, consisting of a tank of 1.2m height, 1.12m wide and a capacity of 1,100 lt. The inside is coated with a reflective Flex polyvinyl chloride sheet. The PMT installed is a *XP1802*.
\item  The detector is placed on the roof of T1 Building , University of San Carlos of Guatemala at a latitude 14 35'18 .06 " , and a longitude 90 33'13 .30" W and an altitude of 1490 m Cut Off Rigidity : approx . 8GV . The tank is filled with comercial demineralized water.
 
 \item The electronic for detector was the original from Bariloche and was running from June 2011 until March 2012 , when it's broke . Later with the visit of Ruben Conde of the University of Puebla began the  developedof a new electronic  system not finished yet.


\end{enumerate}
\
Edgar Cifuentes
Universidad de San Carlos de Guatemala

\end{document}
 Detalles del detector: dimensiones, electrónica, que tipo de PMT se utiliza. Si existe más de un detector notar la cantidad, y si son diferentes lo mismo. Si se encuentran en más de un sitio dentro de un mismo país también hacer esa diferenciación.
* Si hubieran, Histogramas de carga y/o amplitud junto con un pulso típico o promedio de cada detector. 
* Detalle del o los sitios. Los sitios de LAGO estarán ordenados por país, si existiera más de un sitio por país describirlos a todos. Destacar altura, rigidity cut off, latitud y longitud. 
* Hacer notar cuales sitios están operando al día de la fecha de escritura (y desde cuando) cuales instalados y cuales estarán en un futuro. 
-------------
> *1) Detector*:
> - Dimensiones: Desconozco el tamaño del tanque de agua. Solamente tenemos
> información que fueron utilizados 60 garrafones de agua desmineralizada.
> - Electrónica: Exsite acualmente la digitalizadora diseñada en Bariloche
>
> (que fue reparada). Paralelamente, estamos diseñando nuestra electrónica
> para mejorar el rendimiento y no depender de "código fuente" de terceros.
> - El PMT instalado es el *XP1802*. Además, tenemos disponibles (sin
> uso) otros 2 fototubos del modelo *EMI 9530A.* * * *2) Más detectores:*
> No hay
>
> *3) Sitios:*
> Hay solamente un sitio:
> Azotea de Edificio T1, Universidad de San Carlos de Guatemala.
> Ubicación: 
1435'18.06" 90º33'13.30"W. 1490 m.s.n.m.
14º8.06"N y 90𑵉'13.30"W.  1490 m.s.n.m, 
> Rigidity Cut Off: aprox. 8GV (
> http://indico.nucleares.unam.mx/getFile.py/access?
contribId=730&sessionId=39&resId=0&materialId=paper&confId=4
> )
>
> Por el momento, el único sitio está inactivo. Luego de realizar las pruebas
> del estado del agua, se iniciará nuevamente el logging (fecha no definida
> aún).
>
> Iván. 
------------------------------------------------------------------
La compra del tanque, el agua, el blackout y el reflejante fue
1063.63+742+950+175=2930.63
2930.63/7.75 = $378.14
Pero fue necesario comprar algunas cosas de ferreteria y electronica
389+213+165=767
767/7.75=$98.96
Total
378.14+98.96=$477.10
------------------------------------------- -------------------------------------------------
Reporte LAGO-Guatemala
En Guatemala el tanque está colocado en la azotea del edificio que alberga al
Departamento de Física. A una altura de 1490 m.s.n.m, latitud de 14𑵋'18.06"N
y 90𑵉'13.30"W. Desde el Workshop Lago de 2011 el tanque se mantuvo en
funcionamiento hasta el mes de Marzo de 2012, un problema en la electrónica
que estamos tratando de resolver fue la causa del final de la señal. El tubo y
la electrónica en uso es la que Miguel trajo de Bariloche.

Un profesor, Walter Alvarez y un estudiante de licenciatura Rodrigo Ardón son
lo únicos que han estado dandole seguimiento al tanque. Desafortunadamente
perdimos al Profesor Héctor Pérez, quien recibió el entrenamiento en
Bariloche, por problemas de presupuesto en el Departamenteo de Física. Este
tipo de problemas esperabamos resolverlo para este año con la creación de la
Escuela de Ciencias Físicas y Matemáticas que fue aprobada a finales de 2009,
pero a la fecha no se ha hecho efectiva.
En los próximos meses nuestros planes son tratar de hacer funcionar de nuevo
el tanque y para esto hemos estado en comunicación con Miguel. Ademas en la
reciente Reunión que tuvimos en Tuxtla platicamos con Eli Santos y Humberto
Salazar, acerca de empezar a proyectar la instalación de un tanque grande como
los de Sierra Negra en la Frontera de Guatemala y México, el Volcán Tacaná, el
segundo punto mas alto de Guatemla. El tanque estaría alrededor de los 4,000
metros.
Edgar Cifuentes
-------------------------------- 


