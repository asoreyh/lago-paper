\section{Introduction}\label{sec:intro}

Empezamos hablando de los RC en gral

para luego hablar un poquito de observaciones en tierra

y ahi entramos con LAGO


Gamma-ray bursts (GRBs) are the most powerful explosions known in the Universe.
They are sudden emissions of gamma-rays lasting very short time intervals (from
0.1 up to 500 seconds). They were discovered in 1967 by the Vela satellites
that the US government launched to monitor covert nuclear tests from space. In
1997 the BeppoSAX satellite detected an optical afterglow of a GRB.  When it
was analyzed it showed that the redshift of the galaxy that originated the GRB
was at more than 6 billon light years from Earth. This showed GRBs to be
extragalactic events. The astrophysical sources of these events are still
unclear but good candidates are the coalescence of compacts objects for the
short bursts (less than 2 s), and hypernovae, supernovae produced by very
massive stars, for the long bursts (more than 2 s) \cite{Meszaros2006}.

Despite satellites observations like SWIFT, FERMI and others, revealing
some questions about the origin and location of GRBs, other questions remain
unanswered, especially in the high energy region, i.e. the spectrum. Up to now
no ground-based experiment has detected a GRB.

The idea is that when high energy photons from GRBs reach the atmosphere, they
produce cascades that can be detectable at ground level by using WCDs. Instead
of trying to detect this extensive air showers individually, LAGO makes use of the single
particle technique, that is to observe an excess in the counting rates of
secondary particles produced in EAS \cite{Vernetto2000}. The main advantage of
using WCDs compared to other instuments, is their higher sensitivity to
photons, which are 90\% of the secondary particles at ground level for high
energy primary photons.

Although the original plan comprised the detection of the high energy component
of GRBs, recently it has been shown that WCDs can also be used to study the
Solar Modulation (SM) of galactic cosmic rays and other transient effects, by
measuring the variations of the flux of secondary particles at ground level
\cite{PierreAugerCollaboration2011}

The main effect is produced by the solar magnetic field. When the Sun shows a
high activity (intense magnetic field), galactic cosmic rays are deflected,
resulting in a reduction of their flux on Earth. When magnetic fields are less
intense we have a higher flux. So, by measuring the flux of galactic cosmic
rays one can determine indirectly the solar activity. This allows, for
instance, the observation of the eleven year cycle of the sun and also
transient events known as Forbush decreases. These decreases are produced
when a Coronal Mass Ejection (CME) is originated in the Sun, resulting in a
huge mass of plasma sent through the interplanetary medium. Upon reaching
Earth, this plasma perturbs the near space and results in a modification in the
flux of galactic cosmic rays. A rapid diminution of their flux can be
observed (a few percent in a few hours). Then, once the CME leave the earth surroundings,
the galactic cosmic ray flux slowly comes back to its original value, on a time
scale of days.

The LAGO project is a recent collaboration that comes from the association of
Latin American astroparticle researchers. It started in 2005 and it was
designed to survey the high-energy component of GRBs. It is a network of
ground-based WCDs, located at mountain altitudes, above 4500 m.a.s.l., where
the flux of the primary gamma-rays is too low for effective detection by small
area satellite detectors. This collaboration was motivated by the experience of
the Pierre Auger Observatory, and the idea is to install WCD in 9 Latin
American countries: Argentina, Bolivia, Brasil, Colombia, Ecuador, Guatemala,
Mexico, Peru and Venezuela \cite{Allard2008,Allard2009b}. More than 75 people
integrate the LAGO community, keeping a close collaboration with researchers
from IN2P3 of France and INFN in Italy.

LAGO can be an optimal detection network to characterize the SM and transient
events, as it spans over a large area with sites at different heights, latitudes,
longitudes and geomagnetic rigidity cut offs.

